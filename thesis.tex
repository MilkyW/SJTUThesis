% \documentclass[degree=doctor, zihao=-4, review]{sjtuthesis}
\documentclass[degree=master, language=english, fontset=fandol]{sjtuthesis}
% \documentclass[degree=bachelor, openany, oneside]{sjtuthesis}
% \documentclass[degree=course, openright, twoside]{sjtuthesis}
% 选项
%   degree=[doctor|master|bachelor|course], % 必选,学位类型
%   language=[chinese|english], % 可选(默认:chinese),论文的主要语言
%   review,                     % 可选(默认:关闭),盲审模式

% 所有其它可能用到的包都统一放到这里了,可以根据自己的实际添加或者删除。
% \usepackage{sjtuthesis}
\usepackage{kantlipsum}
\usepackage{zhlipsum}

% 逐个导入参考文献数据库
% \addbibresource{bib/thesis.bib}

%# -*- coding: utf-8-unix -*-
% !TEX program = xelatex
% !TEX root = ../thesis.tex
% !TEX encoding = UTF-8 Unicode
%TC:ignore
\title{上海交通大学学位论文 \LaTeX{} 模板示例文档}
\author{某\quad{}某}
\supervisor{某某教授}
% \assisupervisor{某某教授}
\date{2014年12月17日}
\coursename{某某课程}
\department{某某系}
\studentid{0010900990}
\degreeapplied{工学硕士}
\major{某某专业}
\keywords{上海交大, 饮水思源, 爱国荣校}

\englishtitle{A Sample Document for \LaTeX-based SJTU Thesis Template}
\englishauthor{\textsc{Mo Mo}}
\englishsupervisor{Prof. \textsc{Mou Mou}}
% \englishassisupervisor{Prof. \textsc{Uom Uom}}
\englishdepartment{\textsc{Depart of XXX, School of XXX}}
\englishmajor{A Very Important Major}
\englishdate{Dec. 17th, 2014}
\englishdegreeapplied{Master of Engineering}
\englishkeywords{SJTU, master thesis, XeTeX/LaTeX template}
%TC:endignore


\begin{document}

\maketitle
\makeDeclareOriginal
\makeDeclareAuthorization

\frontmatter % 使用罗马数字对前言编号

% 摘要
\include{tex/abstract}

% 目录、插图目录、表格目录
\tableofcontents
% \listoffigures
% \addcontentsline{toc}{chapter}{\listfigurename}     % 将插图目录加入全文目录
% \listoftables
% \addcontentsline{toc}{chapter}{\listtablename}      % 将表格目录加入全文目录
% \listofalgorithms
% \addcontentsline{toc}{chapter}{\listalgorithmname}  % 将算法目录加入全文目录

% \include{tex/symbol} % 主要符号、缩略词对照表

\mainmatter % 使用阿拉伯数字对正文编号

% 正文内容
\zhlipsum[1]
% \include{tex/intro}
% \include{tex/example}
% \include{tex/faq}
% \include{tex/summary}

\appendix % 使用英文字母对附录编号

% 附录内容,本科学位论文可以用翻译的文献替代。
% \include{tex/app_setup}
% \include{tex/app_eq}
% \include{tex/app_cjk}
% \include{tex/app_log}

\backmatter % 文后无编号部分

% 参考资料
% \printbibliography[heading=bibintoc]

% 致谢、发表论文、申请专利、参与项目、简历
% 用于盲审的论文需隐去致谢、发表论文、申请专利、参与的项目
\makeatletter

% \ifsjtu@coursepaper
% \else

%   % "研究生学位论文送盲审印刷格式的统一要求"
%   % http://www.gs.sjtu.edu.cn/inform/3/2015/20151120_123928_738.htm

%   % 盲审删去删去致谢页
%   \ifsjtu@review\relax\else
%     \include{tex/ack}         % 致谢
%   \fi

%   \ifsjtu@bachelor
%     \ifsjtu@english
%     \else
%       % 中文学士学位论文要求在最后有一个英文大摘要,单独编页码,英文学士学位论文不需要
%       \include{tex/end_english_abstract}
%     \fi
%   \else
%     % 盲审论文中,发表学术论文及参与科研情况等仅以第几作者注明即可,不要出现作者或他人姓名
%     \ifsjtu@review\relax
%       \include{tex/pubreview}
%       \include{tex/projectsreview}
%     \else
%       \include{tex/pub}       % 发表论文
%       \include{tex/projects}  % 参与的项目
%       % \include{tex/patents}   % 申请专利
%       \include{tex/resume}    % 个人简历
%     \fi
%   \fi
% \fi

% \makeatother

\end{document}
